%!TEX root = cs610Spring2016_syllabus.tex

\input syllabusprespring
\begin{document}
\MYTITLE{Syllabus}
\MYHEADERS{Syllabus}{}

\vspace*{-.3in}
\subsection*{Course Instructors}

\begin{tabular}{c c}

\begin{minipage}{3.5in}
Oliver Bonham-Carter \\
\noindent Office Location: Alden Hall 104 \\
\noindent Email: \url{obonhamcarter@allegheny.edu} \\
\end{minipage} &

\begin{minipage}{3.5in}
Janyl Jumadinova\\
\noindent Office Location: Alden Hall 105 \\
\noindent Email: \url{jjumadinova@allegheny.edu} \\
\end{minipage} \\

\begin{minipage}{3.5in}
Gregory M.\ Kapfhammer\\
\noindent Office Location: Alden Hall 108 \\
\noindent Email: \url{gkapfham@allegheny.edu} \\
\end{minipage} &

\begin{minipage}{3.5in}
John E.\ Wenskovitch\\
\noindent Office Location: Virginia Tech\\
\noindent Email: \url{jwenskovitch@allegheny.edu} \\
\end{minipage}

\end{tabular}
\vspace*{-.3in}

\subsection*{Instructors' Office Hours}

Please visit the web sites of the course instructors to view their office hours. Using the ``appointment slots'' feature
of Google Calendar, you can select an available meeting time. After picking your time slot, the reserved meeting will
appear in both your Google Calendar and the instructor's.

\vspace*{-.15in}
\begin{itemize}
    \itemsep -.25em
        \item Oliver Bonham-Carter: \url{http://www.cs.allegheny.edu/sites/obonhamcarter/}
        \item Janyl Jumadinova: \url{http://www.cs.allegheny.edu/sites/jjumadinova/}
        \item Gregory M.\ Kapfhammer: \url{http://www.cs.allegheny.edu/sites/gkapfham/}
        \item John E. Wenskovitch: \url{http://www.cs.allegheny.edu/sites/jwenskovitch/}
\end{itemize}

\vspace*{-.25in}
\subsection*{Course Communication}

Throughout the semester, students and faculty will use Slack to support course communication. All students will be
required to integrate notifications from the version control repositories with a specified Slack channel, thereby
allowing all students and the course instructors to observe everyone's progress on their thesis research. Whenever
possible, students are also encouraged to post appropriate questions to a channel in Slack, which is available at
\url{https://CMPSC600Fall2016.slack.com}. Please note that we will use the same Slack team as we did for the Fall 2016
semester.

\vspace*{-.1in}
\subsection*{Course Schedule}

\vspace*{-.1in}
\subsubsection*{CMPSC 610 Course Schedule}

\begin{center}
\begin{tabular}{r|l}
\hline

Before January 24   & Schedule weekly meeting time with your first reader \\
Before February 21  & Schedule thesis defense with Pauline Lanzine \\
Before February 28  & Submit one-page status update on your progress \\
Before March 14     & Submit one-page status update on your progress \\
Before April 4      & Submit unbound and signed thesis to first and second readers by 4 pm \\
April 5--April 25   & Complete oral defense of senior thesis \\
May 2               & Submit final bound and signed thesis to Pauline Lanzine by 4 pm\\

\hline
Entire academic semester & Meet with first reader on a weekly basis \\
Entire academic semester & Participate in discussions in the Slack team \\
\hline
\end{tabular}
\end{center}

\noindent Regardless of your first reader, all students must adhere to the same schedule and due dates. The schedule for
CMPSC 610 may change as the course instructors deem appropriate. Please note that, unless evidence of extenuating
circumstances is presented in writing to all of the course instructors, a student's grade in the course will be reduced
if the stated deadlines are not met. Students who have questions or concerns about these deadlines should talk with
their first reader.

\vspace{-.15in}
\subsection*{Required Textbooks}
\vspace{-.05in}

\noindent{\em On Being a Scientist: A Guide to Responsible Conduct in Research\/} (Third Edition). Committee on Science,
Engineering, and Public Policy, National Academy of Sciences, National Academy of Engineering, and Institute of
Medicine. ISBN: 0309119715, 82 pages, 2009.\\ (References to the textbook are abbreviated as ``OBAS'').

\noindent{\em BUGS in Writing: A Guide to Debugging Your Prose\/} (Second Edition). Lyn Dupr\'e. Addison-Wesley
Professional.  ISBN-10: 020137921X and ISBN-13: 978-0201379211, 704 pages, 1998.\\ (References to the textbook are
abbreviated as ``BIW'').

\noindent{\em Writing for Computer Science} (Second Edition). Justin Zobel. Springer ISBN-10: 1852338024 and ISBN-13:
978-1852338022, 270 pages, 2004. \\ (References to the textbook are abbreviated as ``WFCS'').

\vspace*{-.15in}
\subsection*{Overview of the Grading Policies}

\vspace*{-.15in}
\subsubsection*{Using Email}
\vspace*{-.05in}

Although we will primarily use Slack for class communication, we will sometimes use email to send announcements about
important class matters. It is your responsibility to check your email at least once a day and to ensure that you can
reliably send and receive emails. This class policy is based on the statement about the use of email that appears in
{\em The Compass}, the student handbook.

\vspace*{-.15in}
\subsection*{Honor Code}
\vspace*{-.05in}

The Academic Honor Program that governs the academic program at Allegheny College is described in the Allegheny
Academic Bulletin.  The Honor Program applies to all work that is submitted for academic credit or to meet non-credit
requirements for graduation at Allegheny College.  This includes all work assigned for these classes (e.g., source code,
technical diagrams, and your written content); deliverables that are nearly identical the work of others will be taken
as evidence of violating the Honor Code. All students who have enrolled in the College will work under the Honor
Program.  Each student who has matriculated at the College has acknowledged the following pledge:

\vspace*{-.1in}
\begin{quote}
I hereby recognize and pledge to fulfill my responsibilities, as defined in the Honor Code, and to maintain the
integrity of both myself and the College community as a whole.
\end{quote}
\vspace*{-.3in}

\subsection*{Disability Services}
\vspace*{-.05in}

The Americans with Disabilities Act (ADA) is a federal anti-discrimination statute that provides comprehensive civil
rights protection for persons with disabilities.  Among other things, this legislation requires all students with
disabilities be guaranteed a learning environment that provides for reasonable accommodation of their disabilities.
Students with disabilities who believe they may need accommodations in this class are encouraged to contact Disability
Services at 332-2898.  Disability Services is part of the Learning Commons and is located in Pelletier Library.
Please do this as soon as possible to ensure that approved accommodations are implemented in a timely fashion.

\vspace*{-.1in}
\subsection*{Welcome to an Adventure in Computer Science}

% Moreover, these courses properly position you to conduct ground-breaking work that can have a positive influence on your
% future career and graduate school prospects, the students and faculty in the Department of Computer Science, the
% Allegheny College community, and a broader society that heavily relies on computer hardware and software.

CMPSC 600 affords you the opportunity to pursue independent research in computer science and to ensure that your work
has a positive influence on your future plans, the students and faculty at Allegheny College, and a broader society that
relies heavily on computer hardware and software.  At the start of your senior year, I invite you to pursue this class
with great enthusiasm and vigor.

% \subsection*{Special Needs and Disabilities}
% The Americans with Disabilities Act (ADA) is a federal anti-discrimination
% statute that provides comprehensive civil rights protection for persons
% with disabilities. Among other things, this legislation requires that
% all students with disabilities be guaranteed a learning environment
% that provides for reasonable accommodation of their disabilities.
% If you believe  you have a disability requiring an accommodation,
% please contact the Learning Commons at 332-2898.
%
\end{document}
