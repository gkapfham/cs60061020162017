%!TEX root = cs610Spring2016_syllabus.tex

\input syllabusprespring
\begin{document}
\MYTITLE{Syllabus}
\MYHEADERS{Syllabus}{}

\vspace*{-.3in}
\subsection*{Course Instructors}

\begin{tabular}{c c}

\begin{minipage}{3.5in}
Oliver Bonham-Carter \\
\noindent Office Location: Alden Hall 104 \\
\noindent Email: \url{obonhamcarter@allegheny.edu} \\
\end{minipage} &

\begin{minipage}{3.5in}
Janyl Jumadinova\\
\noindent Office Location: Alden Hall 105 \\
\noindent Email: \url{jjumadinova@allegheny.edu} \\
\end{minipage} \\

\begin{minipage}{3.5in}
Gregory M.\ Kapfhammer\\
\noindent Office Location: Alden Hall 108 \\
\noindent Email: \url{gkapfham@allegheny.edu} \\
\end{minipage} &

\begin{minipage}{3.5in}
John E.\ Wenskovitch\\
\noindent Office Location: Virginia Tech\\
\noindent Email: \url{jwenskovitch@allegheny.edu} \\
\end{minipage}

\end{tabular}
\vspace*{-.3in}

\subsection*{Instructors' Office Hours}

Please visit the web sites of the course instructors to view their office hours. Using the ``appointment slots'' feature
of Google Calendar, you can select an available meeting time. After picking your time slot, the reserved meeting will
appear in both your Google Calendar and the instructor's.

\vspace*{-.15in}
\begin{itemize}
    \itemsep -.25em
        \item Oliver Bonham-Carter: \url{http://www.cs.allegheny.edu/sites/obonhamcarter/}
        \item Janyl Jumadinova: \url{http://www.cs.allegheny.edu/sites/jjumadinova/}
        \item Gregory M.\ Kapfhammer: \url{http://www.cs.allegheny.edu/sites/gkapfham/}
        \item John E. Wenskovitch: \url{http://www.cs.allegheny.edu/sites/jwenskovitch/}
\end{itemize}

\vspace*{-.25in}
\subsection*{Course Communication}

Throughout the semester, students and faculty will use Slack to support course communication. All students will be
required to integrate notifications from the version control repositories with a specified Slack channel, thereby
allowing all students and the course instructors to observe everyone's progress on their thesis research. Whenever
possible, students are also encouraged to post appropriate questions to a channel in Slack, which is available at
\url{https://CMPSC600Fall2016.slack.com}. Please note that we will use the same Slack team as we did for the Fall 2016
semester.

\vspace*{-.1in}
\subsection*{Course Schedule}

\vspace*{-.1in}
\subsubsection*{CMPSC 610 Course Schedule}
\vspace*{-.1in}

\begin{center}
\begin{tabular}{r|l}
\hline

Before January 24  & Schedule weekly meeting time with your first reader \\
Before February 21 & Schedule thesis defense with Pauline Lanzine \\
February 28        & Submit three-paragraph status update on your progress \\
March 14           & Submit three-paragraph status update on your progress \\
April 4            & Submit unbound and signed thesis to first and second readers by 4 pm \\
April 5--April 25  & Complete oral defense of senior thesis \\
May 2              & Submit final bound and signed thesis to Pauline Lanzine by 4 pm\\

\hline
Entire academic semester & Meet with first reader on a weekly basis \\
Entire academic semester & Participate in discussions in the Slack team \\
\hline
\end{tabular}
\end{center}

\noindent Regardless of your first reader, all students must adhere to the same course schedule and due dates. The
schedule for CMPSC 610 may change as the course instructors deem appropriate. Please note that, unless evidence of
extenuating circumstances is presented in writing to all of the course instructors, a student's grade in the course will
be reduced if the stated deadlines are not met. Students who have questions or concerns about these deadlines should
talk with their first reader.

\vspace{-.15in}
\subsection*{Required Textbooks}
\vspace{-.05in}

\noindent{\em On Being a Scientist: A Guide to Responsible Conduct in Research\/} (Third Edition). Committee on Science,
Engineering, and Public Policy, National Academy of Sciences, National Academy of Engineering, and Institute of
Medicine. ISBN: 0309119715, 82 pages, 2009.
% \\ (References to the textbook are abbreviated as ``OBAS'').

\noindent{\em BUGS in Writing: A Guide to Debugging Your Prose\/} (Second Edition). Lyn Dupr\'e. Addison-Wesley
Professional.  ISBN-10: 020137921X and ISBN-13: 978-0201379211, 704 pages, 1998.

% \\ (References to the textbook are abbreviated as ``BIW'').

\noindent{\em Writing for Computer Science} (Second Edition). Justin Zobel. Springer ISBN-10: 1852338024 and ISBN-13:
978-1852338022, 270 pages, 2004.

% \\ (References to the textbook are abbreviated as ``WFCS'').

\vspace*{-.15in}
\subsection*{Overview of the Grading Policies}

Final grades are determined after the entire faculty of the Department of Computer Science, not just your course
instructor for CMPSC 610, review and discuss all of the submitted deliverables. Your grade in CMPSC 610 will be based on
a combination of the following activities and deliverables. Percentages are not given because we recognize that the
senior thesis experience differs from one student to the next and that there are many variables, such as the nature of
the project and the availability of external resources, that can influence the relative importance of these criteria.
However, it is important to note that a large percentage of your grade depends upon your class participation, final
senior thesis chapters, and the oral defense of your thesis.

\vspace*{-.05in}

\begin{itemize}
  \itemsep -.25em

  \item {\bf Class Participation}: This includes meeting regularly with your first reader. Although the exact details
    about frequency and length of each meeting must be established with your first reader, you should adhere to the
    previously stated schedule. Additionally, this also requires regular contributions, in the form of questions and
    comments, to the course's Slack site. You must hold productive meetings with your first reader, who will report on
    the quality of your meetings when the department's faculty meet to assign grades.  Students are expected to come to
    each meeting with a status update on their progress and a meeting agenda.  Students should conclude each meeting by
    listing the tasks that they want to complete before the next meeting. Evidence of regular meetings must be submitted
    to the course instructor.

  \item {\bf Status Updates}: Twice during the semester a student must submit a three-paragraph update on their progress
    during CMPSC 610. The first paragraph should review your completed work, the second should explain the work on which
    you are currently focusing, and the final paragraph should outline what you intend to have finished by the next
    status update.

  \item {\bf Course Repositories}: This involves students using, at minimum, two version control repositories to
    store (i) their thesis proposal and written chapters and (ii) any relevant source code and data.  Both of your
    readers must have administrative access to your repositories.

  \item {\bf Written Thesis.} In consultation with your first reader and in accordance with the stated deadlines, you
    must work out a schedule for completion of your thesis research and your written document. All senior theses are
    due, properly formatted and signed (but not bound), on the stated due date.  Working closely with your first reader,
    you must produce a thesis that both follows the department's style and adheres to professional standards of writing.
    Your grade in CMPSC 610 will be reduced if you fail to submit your unbound thesis on time.

    Following your defense, you must submit the bound copy of your senior thesis by the aforementioned due date.  This
    document must incorporate any changes that were requested by your first and second reader. Seniors who have not
    delivered the signed and bound copies of their senior thesis by the stated deadline will receive an
    incomplete and will not graduate.

  \item{\bf Thesis Defense.} The standards for this presentation are the same as for the proposal defense --- you must
    give a ten to fifteen minute presentation supported by polished slides and adhere to all of the other stated
    requirements for this deliverable.  Part of your grade for this will depend on how well you are able to discuss
    aspects of your thesis, including implications of your work, connections between your research and other areas of
    computer science, and possible extensions or improvements of your research ideas.  You are expected to work with
    your first reader in preparing your oral defense.  Your grade in CMPSC 610 will be reduced if you do not schedule or
    conduct your thesis defense by the stated deadlines.

    Unless there are severe extenuating circumstances, students are not allowed to reschedule their defense once they
    have a confirmed date from Pauline Lanzine. To schedule your thesis defense, please check the Google Calendar of
    your first and second readers and come to Pauline's office with three dates and times that fit into your schedule
    and the schedules of your readers. Do not suggest dates and times that conflict with the schedules of your readers!

\end{itemize}

% \vspace*{-.15in}
% \subsubsection*{Using Email}
% \vspace*{-.05in}

% Although we will primarily use Slack for class communication, we will sometimes use email to send announcements about
% important class matters. It is your responsibility to check your email at least once a day and to ensure that you can
% reliably send and receive emails. This class policy is based on the statement about the use of email that appears in
% {\em The Compass}, the student handbook.

\vspace*{-.25in}
\subsection*{Honor Code}
\vspace*{-.1in}

The Academic Honor Program that governs the academic program at Allegheny College is described in the Allegheny
Academic Bulletin. The Honor Program applies to all work that is submitted for academic credit or to meet non-credit
requirements for graduation at Allegheny College. This includes all work assigned for these classes (e.g., source code,
technical diagrams, and your written content); deliverables that are nearly identical the work of others will be taken
as evidence of violating the Honor Code.

% All students who have enrolled in the College will work under the Honor
% Program.  Each student who has matriculated at the College has acknowledged the following pledge:

% \vspace*{-.1in}
% \begin{quote}
% I hereby recognize and pledge to fulfill my responsibilities, as defined in the Honor Code, and to maintain the
% integrity of both myself and the College community as a whole.
% \end{quote}
% \vspace*{-.3in}

\vspace*{-.1in}
\subsection*{Disability Services}
\vspace*{-.05in}

The Americans with Disabilities Act (ADA) is a federal anti-discrimination statute that provides comprehensive civil
rights protection for persons with disabilities.  Among other things, this legislation requires all students with
disabilities be guaranteed a learning environment that provides for reasonable accommodation of their disabilities.
Students with disabilities who believe they may need accommodations in this class are encouraged to contact Disability
Services at 332-2898.  Disability Services is part of the Learning Commons and is located in Pelletier Library.
Please do this as soon as possible to ensure that approved accommodations are implemented in a timely fashion.

\vspace*{-.1in}
\subsection*{Welcome to an Adventure in Computer Science}
\vspace*{-.1in}

% Moreover, these courses properly position you to conduct ground-breaking work that can have a positive influence on your
% future career and graduate school prospects, the students and faculty in the Department of Computer Science, the
% Allegheny College community, and a broader society that heavily relies on computer hardware and software.

CMPSC 610 affords you the opportunity to pursue independent research in computer science and to ensure that your work
has a positive influence on your future plans, the students and faculty at Allegheny College, and a broader society that
relies heavily on computer hardware and software. As you finish your senior year, I invite you to pursue this class with
great enthusiasm and vigor.

% \subsection*{Special Needs and Disabilities}
% The Americans with Disabilities Act (ADA) is a federal anti-discrimination
% statute that provides comprehensive civil rights protection for persons
% with disabilities. Among other things, this legislation requires that
% all students with disabilities be guaranteed a learning environment
% that provides for reasonable accommodation of their disabilities.
% If you believe  you have a disability requiring an accommodation,
% please contact the Learning Commons at 332-2898.
%
\end{document}
